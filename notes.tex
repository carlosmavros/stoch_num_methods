% Created 2021-12-02 Thu 11:44
% Intended LaTeX compiler: pdflatex
\documentclass[12pt,a4paper]{article}
                \usepackage[utf8]{inputenc}
                \usepackage[greek, english]{babel}
                \usepackage{caption}
                \usepackage{mathtools}
                \usepackage{amsmath}    %math
                \usepackage{amsfonts}   %math fonts
                \usepackage{amssymb}
                \usepackage{amsthm}
                \usepackage{enumitem}
                \usepackage{geometry}   %page dimensions
                \usepackage[table,xcdraw]{xcolor}
                \usepackage{euscript}[mathscr]
                \usepackage[table,xcdraw]{xcolor}
                \newtheorem{theorem}{Theorem}[section]
                \newtheorem{corollary}{Corollary}[section]
                \newtheorem{lemma}[theorem]{Lemma}
                \theoremstyle{definition}
                \numberwithin{equation}{section}
                \setlist[itemize,1]{label=$\diamond$}
                \usepackage{hyperref}
                \hypersetup{colorlinks,citecolor=black,filecolor=black,linkcolor=black,urlcolor=black}
                \newtheorem{exercise}{Άσκηση}
                \newtheorem{paradeigma}{Παράδειγμα}
                \newtheorem{protasi}{Πρόταση}
                \newtheorem{orismos}{Ορισμός}
                \newtheorem{theorima}{Θεώρημα}
                \newtheorem{limma}{Λήμμα}
                \selectlanguage{greek}

\selectlanguage{greek}
\author{\textgreek{Κάρλος Μαύρος - ΣΕΜΦΕ ΕΜΠ}}
\date{\today}
\title{\textgreek{Στοχαστικές Αριθμητικές Μέθοδοι και Εφαρμογές}\\\medskip
\large \textgreek{Διδάσκων: Σαμπάνης Σ.}}
\hypersetup{
 pdfauthor={\textgreek{Κάρλος Μαύρος - ΣΕΜΦΕ ΕΜΠ}},
 pdftitle={\textgreek{Στοχαστικές Αριθμητικές Μέθοδοι και Εφαρμογές}},
 pdfkeywords={},
 pdfsubject={},
 pdfcreator={Emacs 27.2 (Org mode 9.5)}, 
 pdflang={English}}
\begin{document}

\maketitle
\selectlanguage{greek}

\section{ΕΙΣΑΓΩΓΗ}
\label{sec:org1125d53}

\begin{itemize}
\item \textlatin{MCMC (Markov Chain Monte Carlo).}
\item \textlatin{Langevin Stochastic DEs}: βλέπουμε τις στοχαστικές λύσεις σαν στοχαστικές διαφορικές εξισώσεις.
\item Βελτιστοποίηση μη κυρτών συναρτήσεων σε χώρους μεγάλων διαστάσεων.
\end{itemize}



Στα  \textlatin{gradient methods \& stochastic gradient methods} υπάρχουν 2 σχολές:
\begin{itemize}
\item Επιχειρησιακή έρευνα (κυρτές συναρτήσεις).
\item Μέσα από την θεωρία του \textlatin{Stochastic Approximation:} χρησιμοποιεί ΔΕ σαν εργαλεία
\end{itemize}

Το \textlatin{stochastic gradient methods} δεν είναι πραγματικά στοχαστικές (υπολογίζουμε απλώς μια μέση τιμή)
Οι \textlatin{stochastic gradient methods} είναι ένα υποσύνολο της θεωρίας \textlatin{Stochastic Approximation}, η οποία
χρησιμοποιεί πραγματικά στοχαστικά εργαλεία (έχουμε μέσα στοχαστικές διαδικασίες).

Εργαλεία:
\begin{itemize}
\item σ.β. σύγκλιση
\item σύγκλιση με πιθανότητα
\item \textlatin{Ito's formula} (σημαντικό) διαχωρίζει το δυναμικό σύστημα τ.ω. να μπορούμε να αναγνωρίζουμε ποια
είναι τα \textlatin{martingales.} Βλέπω τις τάσεις του δυναμικού συστήματος.
\end{itemize}

\subsection{ΔΙΑΔΙΚΑΣΤΙΚΑ ΜΑΘΗΜΑΤΟΣ}
\label{sec:org4e56a43}
\begin{itemize}
\item θα γίνει εξέταση
\item βιβλιογραφία:
\begin{itemize}
\item (Θεωρία Πιθανοτήτων) \textlatin{David Williams} : \textlatin{Probability with martingales}
\item (Στοχαστικές Διαδικασίες/Ανάλυση) Καραντζάς \& \textlatin{Steven}
\end{itemize}
\end{itemize}

\newpage
\section{Μάθημα 1}
\label{sec:org8c16069}

\subsection{Εισαγωγή}
\label{sec:orgaa9c138}

\begin{itemize}
\item Ονομάζουμε σύνολο κάθε συλλογή αντικειμένων όπου η διάταξη δεν έχει σημασία.
\item Κάθε μέρος του συνόλου ονομάζεται υποσύνολο του συνόλου.
\item Έστω \(\Omega\) σύνολο, τότε το δυναμοσύνολο του \(\Omega\) είναι το σύνολο όλων
των υποσυνόλων του \(\Omega\) και το συμβολίσουμε \(\mathcal{P}(\Omega)\).
\item Για κάθε σύνολο \(\Omega = \{1,2,\dots, n\}\) όπου \(n\in \mathbb{N}\) το \(\mathcal{P}(\Omega)\) έχει \(2^n\) στοιχεία.
\end{itemize}

\begin{paradeigma}
    $$\Omega = \{ 1,2,3,\} \quad \mathcal{P}(\Omega) = \{\emptyset, \Omega, \{1\}, \{2\}, \{3\}, \{1,2,\}, \{1,3\}, \{2,3\}\} = 2^\Omega$$
\end{paradeigma}

\begin{orismos}{(σ-άλγεβρα)}: ονομάζουμε σ-άλγεβρα $\mathcal{F}$ ενός συνόλου $\Omega$ κάθε σύνολο υποσυνόλων του $\Omega$
με τις εξής ιδιότητες:

1. $\emptyset \in \mathcal{F}$
2. $A \in \mathcal{F} \Rightarrow A^c \in \mathcal{F}$
3. $A_1,A_2,\dots \in \mathcal{F} \Rightarrow \bigcup_{i}A_i \in \mathcal{F}$

\end{orismos}

\begin{paradeigma}
    Τετριμμένη σ-άλγεβρα: $\mathcal{F} = \{\emptyset, \Omega\}$\\
\end{paradeigma}

\begin{paradeigma}
    Για κάθε $A\subset \Omega$ μπορώ να φτιάξω την $\mathcal{F} = \{\emptyset, A, A^c, \Omega\}$
    που είναι σ-άλγεβρα.
\end{paradeigma}

\end{paradeigma}
    Αν έχω μια αριθμήσιμη συλλογή από pairwise disjoint sets \(A_1,A_2,\dots\)
    δηλαδή \(A_i \cap A_j = \emptyset \: \forall i \neq j\) και \(\bigcup_i A_i = \Omega\)
    διαμέριση του \(\Omega\), τότε
    $$\mathcal{F} = \{\emptyset, \Omega, A_1, A_2, \dots, \text{όλες τις πιθανές ενώσεις των }\: A_i\}$$

Αν έχω μια διαμέριση μπορώ να πάρω όλα τα συμπληρώματα με μόνο ενώσεις, δηλαδή αν είχα τα σύνολα διαμέρισης
\(A_1,A_2,A_3,A_4\) θα είχαμε \((A_1 \cup A_2)^c = A_3 \cup A_4\)
\end{paradeigma}
\newpage

\section{Μάθημα 2}
\label{sec:orge7f23df}

\subsection{Στοιχεία Θεωρίας Πιθανοτήτων}
\label{sec:orgf559159}

\begin{orismos}{(Παραγόμενη σ-άλγεβρα)}:
Αν $\mathcal{A}$ είναι μια συλλογή υποσυνόλων του $\Omega$,
τότε μπορούμε να βρούμε πάντοτε μια σ-άλγεβρα που να περιέχει το $\mathcal{A}$, η οποία είναι
το δυναμοσύνολο $\mathcal{P}(\Omega)$.\\
Παίρνωντας την τομή όλων των σ-αλγεβρών που περιέχουν το $\mathcal{A}$ καταλήγουμε
στην παραγόμενη σ-άλγεβρα (ή ελάχιστη σ-άλγεβρα).

$$\sigma(\mathcal{A}) = \bigcap_{\mathcal{A} \in \mathcal{F}}  \mathcal{F} \quad\quad\quad \text{ όπου κάθε } \mathcal{F} \: \text{σ-άλγεβρα}$$
\end{orismos}
\textbf{Σημαντικές Ιδιότητες}: Έστω \(\mathcal{F}\) μια σ-άλγεβρα ενός συνόλου \(\Omega\)
\begin{itemize}
\item \(A,B \in \mathcal{F} \Rightarrow A\cap B \in \mathcal{F}\)
\item Η τομή δύο ή περισσοτέρων σ-αλγεβρών είναι επίσης σ-άλγεβρα.
\end{itemize}


\begin{orismos}{(Borel σ-άλγεβρα)}\\
Ονομάζουμε σ-άλγεβρα \textlatin{Borel} (ή \textlatin{Borel} σύνολα), συμβ. $\mathcal{B}({\mathbb{R^d})$ (στο $\mathbb{R^d},\: d=1,2,\dots$), την ελάχιστη
σ-άλγεβρα (παραγόμενη) που περιέχει όλα τα ανοικτά υποσύνολα του $\mathbb{R}^d$.\\
\end{orismos}

\begin{protasi}
Η σ-άλγεβρα \textlatin{Borel} είναι η μικρότερη σ-άλγεβρα που περιέχει τα διαστήματα της μορφής
$$(-\infty,\alpha] \quad \alpha \in \mathbb{Z}$$

\textbf{Απόδειξη}\\
Έστω $\mathcal{O}$ το σύνολο όλων των ανοικτών συνόλων του $\mathbb{R}$, τότε $\sigma(\mathcal{O}) = \mathcal{B}(\mathbb{R})$.\\
Έστω $\mathcal{D}$ το σύνολο όλων των διαστημάτων της μορφής $(-\infty,\alpha] \quad \alpha \in \mathbb{Z}$.\\
Έστω τώρα μια φθίνουσα ακολουθία $\{\alpha_k\}_{k\geq 1} \subset \mathbb{Z} $ ρητών αριθμών τ.ω. $\alpha_k \downarrow \alpha \in \mathbb{R}$ και έστω μια αύξουσα
ακολουθία $\{\beta_k\}_{k\geq 1} \subset \mathbb{Z}$ τ.ω. $\beta_k \upnarrow b \in \mathbb{R}$. Συνεπώς μιας και
$$ (\alpha,\beta) = \bigcup_{n=1}^{\infty} \big( (-\infty,\beta_n]\cap(-\infty,\alpha_n]^c\big)$$
Καταλήγουμε στο ότι το $(\alpha,\beta)$ ανήκει στην $\sigma(\mathcal{D})$ για κάθε $\alpha,\beta \in \mathbb{R}$ και άρα έχουμε $\mathcal{B}(\mathbb{R})\subset \sigma(\mathcal{D})$.\\
Από την άλλη έχουμε $\sigma(\mathcal{D}) \subset \mathcal{B}(\mathbb{R}$ αφού τα διαστήματα στο $\mathcal{D}$ μπορούμε να τα δούμε ως συμπληρώματα ανοικτών διαστημάτων, συνεπώς η
ελάχιστη σ-άλγεβρα που περιέχει τέτοια ανοικτά υποσύνολα/διαστήματα θα είναι υποσύνολο της ελάχιστης σ-άλγεβρας που περιέχει όλα τα ανοικτά υποσύνολα του $\mathbb{R}$.\qed
\end{protasi}

\textbf{\textbf{Σημείωση}}: Γενικά αν \(A \subset B \Rightarrow \sigma(A) \subset \sigma(B)\) και αν \(\mathcal{F}\) είναι σ-άλγεβρα τότε\\
\(\sigma (\mathcal{F}) = \mathcal{F}\)

\begin{itemize}
\item Μονοσύνολα της μορφής \(\{a\}\) όπου \(a \in \mathbb{R}\) ανήκουν στην \(\mathcal{B}(\mathbb{R})\).
\item \(\mathbb{N},\mathbb{Q}, \mathbb{R}\setminus \mathbb{Q} \in \mathcal{B}(\mathb{R})\).
\end{itemize}

\begin{orismos}(Μετρήσιμο σύνολο)}\\
Έστω $\mathcal{F} μια σ-άλγεβρα. Το σύνολο $A\in \mathcal{F}$ λέγεται $\mathcal{F}-\textμετρήσιμο}$  ($\mathcal{F}-\textlatin{measurable)}$.
\end{orismos}

\begin{orismos}{(Μερησιμος χώρος)}\\
Έστω $\mathcal{F}$ μια σ-άλγεβρα υποσυνόλων ενός συνόλου $\Omega$. Τότε το ζεύγος $(\Omega,\mathcal{F})$ ονομάζεται
μετρήσιμος χώρος (\textlatin{measurable space)}
\end{orismos}

\begin{orismos}{(Μετρήσιμη συνάρτηση)}\\
Έστω $\Omega$ ένα μη-κενό σύνολο, $\mathcal{F}$ μια σ-άλγεβρα του $\Omega$ και $f:\Omega \mapsto \mathbb{R}^n$. Η συνάρτηση
$f$ ονομάζεται $\mathcal{F}$-μετρήσιμη (ή απλώς μετρήσιμη) αν για κάθε σύνολο \textlatin{Borel} $B$, δηλαδή $B\in\mathcal{B}(\mathbb{R^n})$
$$ f^{-1}(B) = \{ \omega \in \Omega \: : \: f(\omega) \} \in \mathcal{F}$$
\end{orismos}

\textbf{\textbf{Τα παρακάτω είναι ισοδύναμα}}
\begin{itemize}
\item Η \(f\) είναι μετρήσιμη
\item Για κάθε ανοικτό σύνολο \(A\subset \mathbb{R}^n\) ισχύει \(f^{-1}(A) \in \mathcal{F}\).
\item Για κάθε κλειστό σύνολο \(B\subset \mathbb{R}^n\) ισχύει \(f^{-1}(B) \in \mathcal{F}\).
\end{itemize}

\textbf{\textbf{Σημείωση}}: Η μετρησιμότητα (measurability) μιας συνάρτησης εξαρτάται από το πόσο μεγάλη είναι η σ-άλγεβρα.
\begin{itemize}
\item Αν \(\mathcal{F} = \{ \emptyset, \Omega\}\) τότε μετρήσιμες είναι μόνο οι σταθερές συναρτήσεις, δηλ. \(f(\omega) = c \in \mathbb{R}, \forall \omega \in \Omega\).\\
Αν \(B\in \mathcal{B}(\mathbb{R})\), όπου \(B\) ανοικτό σύνολο, τότε
$$ f^{-1}(B)= \begin{cases}
    \emptyset \quad , c \notin B \\
    \Omega \quad , c \in B
    \end{cases}$$

\item Αν \(A\subset\Omega\), \(\mathcal{F}=\{\emptyset, A, A^c,\Omega \}\) τότε:

$$ \mathbf{1}_A(\omega) = \begin{cases} 1 \quad , \omega \in A\\
  0 \quad , \omega \in A^c
  \end{cases}$$

$$f(\omega) = \begin{cases} c_1\quad , \omega \in A\\
  c_2 \quad , \omega \in A^c
  \end{cases}$$

Γιατί; Έστω \(B \in \mathcal{B}(\mathbb{R})\) τότε
$$ f^{-1}(B) = \begin{cases}
  \emptyset \quad , 0,1 \notin B\\
  A \quad , 1 \in B\\
  A^c \quad , 0 \in B \\
  \Omega \quad, 0,1, \in B
  \end{cases}$$

\pagebreak
\end{itemize}

\section{Μάθημα 3}
\label{sec:org242e967}

\subsection{Ιδιότητες μετρήσιμων συναρτήσεων}
\label{sec:org7768995}
\begin{enumerate}
\item Οι δείκτριες συναρτήσεις ενός μετρήσιμου συνόλου είναι μετρήσιμες ( \(A \in \mathcal{F} \Rightarrow \mathbf{1}_A \text{ είναι } \mathcal{F}\text{-μετρήσιμη}\) )
\item Το άθροισμα, η διαφορά, το γινόμενο και το πηλίκο (όπου ορίζεται) μετρήσιμων συναρτήσεων είναι μετρήσιμα.
\item Το μέγιστο και το ελάχιστο δύο ή περισσοτέρων (πεπερασμένων) μετρήσιμων συναρτήσεων είναι μετρήσιμα.
\item Το όριο (όταν υπάρχει) μιας ακολουθίας μετρήσιμων συναρτήσεων είναι μετρήσιμο όπως επίσης το \(\liminf\) και το \(\limsup\).
\item Το \(\sup\) και το \(\inf\) μιας ακολουθίας μετρήσιμων συναρτήσεων είναι μετρήσιμα.
\item Η σύνθετη συνάρτηση \(g \circ f\) μιας μετρήσιμης συνάρτησης \(f\) με μια συνεχή συνάρτηση \(g\) είναι μετρήσιμη συνάρτηση.
\end{enumerate}

Ως αποτέλεσμα, οι συναρτήσεις \(f^+\) και \(f^-\) οι οποίες ορίζονται ως
$$ f^+(x) = \max (f(x),0) \qquad f^-(x) = - \min(f(x),0)$$
είναι μετρήσιμες αν η \(f\) είναι μετρήσιμη.\\

(Για τα παραπάνω δεν θα κάνουμε απόδειξη σε αυτό το μάθημα, τα χρησιμοποιούμε ελεύθερα στις ασκήσεις και στην εξέταση με απλή αναφορά τους)

\subsection{Θεωρία Μέτρου}
\label{sec:orgce98211}

\begin{orismos} Έστω $(\Omega, \mathcal{F})$ είναι μετρήσιμος χώρος και έστω $\mu: \mathcal{F} \rightarrow \mathbb{R} \cup \{+\infty\}$
είναι μια συάρτηση. Τότε, η $\mu$ ονομάζεται \textbf{μέτρο} αν:
\begin{enumerate}
\item Για όλα τα $A \in \mathcal{F}$ έχουμε $\mu(A) \geq 0$.
\item $\mu(\emptyset)=0$.
\item Αν τα σύνολα $A_1,A_2, \dots \in \mathcal{F}$ είναι ξένα μεταξύ τους ανά δύο τότε $\mu\big( \bigcuo_{i=1}^\infty A_i\big) = \sum_{i=1}^\infty \mu(A_i)$ (αριθμήσιμη προσθετικότητα)
\end{enumerate}\\
\end{orismos}

\begin{orismos} Ως \textbf{μέτρο πιθανόητας} ορίζουμε σε μία σ-άλγεβρα $\mathcal{F}$ ενός συνόλου $\Omega$, μια συνάρτηση
$P: \mathcal{F} \rightarrow [0,1]$ η οποία ικανοποιέι τις ιδιότητες ενός μέτρου και $P(\Omega)=1$.\\
\end{orismos}

\begin{orismos} Ονομάζουμε \textbf{χώρο πιθανότητας} την τριάδα $(\Omega, \mathcal{F}, P)$, όπου $\Omega$ είναι ένα σύνολο (που συχνά ονομάζεται δειγματοχώρος/sample space), $\mathcal{F}$ είναι μια σ-άλγεβρα υποσυνόλων του $\Omega$ και $P:\mathcal{F}\rightarrow [0,1]$ είναι ένα μέτρο πιθανότητας.\\
\end{orismos}

\subsubsection{Ιδιότητες μέτρων πιθανότητας}
\label{sec:org5f68cc4}
Θεωρούμε τον χ.π. \((\Omega, \mathcal{F},P)\). Τότε ισχύουν τα εξής:
\begin{enumerate}
\item \textlatin{(coutable subadditivity)}. Για κάθε \(A_1,A_2,\dots \in \mathcal{F}\) έχουμε \(P\big( \bigcup_{i\geq 1} \big) \leq \sum_{i\geq 1} P(A_i)\).
\item \textlatin{(monotonicity)}. Για κάθε \(A,B \in \mathcal{F}\) με \(A\subset B\) έχουμε \(P(A)\leq P(B)\).
\item \textlatin{(continuity)}. Έστω \(A_1 \subset A_2 \subset \dots\) όπου \(A_1, A_2,\dots \in \mathcal{F}\) μια αύξουσα ακολουθία ενδεχομένων, τότε \(\lim_{n\rightarrow \infty}P(A_n) = P \big( \bigcup_{n=1}^\infty A_n \big)\)
\end{enumerate}

\textbf{Σημείωση}: Οι παραπάνω ιδιότητες ισχύοτν για οποιοδήποτε μέτρο.\\

Πιο κάτω παραθέτουμε μια απόδειξη της Ιδιότητας 3.

\begin{align*}
P\big( \bigcup_{n=1}^\infty A_n \big) &= P\big( \bigcup_{n=1}^\infty (A_n \setminus A_{n-1}\big)\\
\text{(countable additivity) } &= \sum_{n=1}^{\infty} P(A_n \setminus A_{n-1}) \\
&= \lim_{n\rightarrow \infty} \sum_{i=1}^n P(A_i\setminus A_{i-1})\\
\text{(countable additivity) } &= \lim_{n \rightarrow \infty} P\big( \bigcup_{i=1}^n(A_n \setminus A_{n-1} ) \big)\\
&= \lim_{n \rightarrow \infty} P(A_n)
\end{align*}

\textbf{Ιδιότητα} (πηγάζει από την 3) Έστω \(A_1 \supset A_2 \supset \dots\) \textlatin{(contracting sequence of events)}, τότε ισχύει ότι:
$$\lim_{n\rightarrow \infty} P(A_n) = P\big( \bigcap_{n=1}^\infty A_n\big)$$
\pagebreak

\begin{orismos}
Έστω $(\Omega, \mathcal{F})$ μετρήσιμος χώρος και $\mu: \mathcal{F} \rightarrow \mathbb{R}\cup\{+\infty\}$ είναι ένα μέτρο. Τότε
ονομάζουμε αυτό το μέτρο:
\begin{enumerate}
\item \textbf{πεπερασμένο}, αν $\mu(\Omega) < \infty$.
\item \textbf{σ-πεπερασμένο}, αν υπάρχει μια ακολουθία $\{A_n\}_{n\geq 0}$ στοιχείων της $\mathcal{F}$ τέτοια
ώστε $\mu(A_n) < \infty \: \forall n \in \mathbb{R}$ και $\bigcup_{n\geq 1} A_n = \Omega$
\end{enumerate}
\end{orismos}

\subsubsection{Θεώρημα Καραθεοδωρή (εκτός ύλης)}
\label{sec:orgf2e7b1d}

\begin{orismos} Έστω $\Omega$ είναι ένα μη-κενό σύνολο. Ονομάζουμε ένα σύνολο υποσυνόλων
$\mathcal{G}$ του $\Omega$ ως \textbf{π-σύστημα (ή άλγεβρα)} αν είναι κλειστό ως προς τις πεπερασμένες τομές,
δηλαδή:
$$G_1,G_2 \in \mathcal{G} \Rightarrow G_1 \cap G_2 \in \mathcal{G}$$
\end{orismos}

\begin{protasi}
Αν δύο μέτρα πιθανότητας συμπίπτουςν σε ένα π-σύστημα, τότε συμπίπτουν και στην σ-άλγεβρα που παράγεται
από το π-σύστημα.
\end{protasi}

\begin{theorima} \textbf{Caratheodory's Extension Theorem}\\
Έστω $\Omega$ έιναι ένα σύνολo, $\mathcal{G}$ ένα π-σύστημα του $\Omega$ και $\mathcal{F}= \sigma(\mathcal{G})$. Αν το $\mu_0$ είναι μια αριθμήσιμα προσθετική συνάρτηση από το $\mathcal{G}$ στο $[0,+\infty]$, δηλ. $\mu_0 : \mathcal{G} \rightarrow \mathbb{R}_+ \cup \{+\infty\}$.\\
Τότε υπάρχει μέτρο στο $(\Omega, \mathcal{F})$ τέτοιο ώστε
$$\mu(A) = \mu_0 (A) \quad \forall A \in \mathcal{G}$$
Αν μάλιστα $\mu_0(\Omega) < \infty$, τότε υπάρχει μοναδικό τέτοιο μέτρο $\mu$.
\end{theorima}

\begin{paradeigma}
Μέτρο \textlatin{Lebesgue} στο $(\Omega, \mathcal{F}) = ((0,1], \mathcal{B}( (0,1]))$. Θεωρούμε όλα εκείνα τα υποσύνολα του $\Omega$ τα οποία μπορούν
να γραφτούν ως πεπερασμένες ενώσεις των διαστημάτων $(a_1,b_1], \dots (a_n,b_n]$ όπου $n \in \mathbb{N}$ και $0 < a_1 \leq b_1 \leq \dots \leq a_n \leq b_n \leq 1$.
Αν $\mathcal{G}$ είναι το π-σύστημα (άλγεβρα) που περιέχει όλα αυτά τα υποσύνολα, τότε $\mathcal{F} = \sigma(\mathcal{G}) = \mathcal{B}((0,1])$\\

Ορίζουμε επίσης για κάθε σύνολο $G \in \mathcal{G}$, τη συνάρτηση
$$\mu_0(G) = \sum_{k\leq r} (b_k - a_k)$$
όπου αυτό το $G$ είναι $G=(a_1,b_1]\cup\dots\cup (a_r,b_r]$ και $r \leq n$. Έτσι η $\mu_0$ είναι καλώς ορισμένη (well-defined) και είναι αριθμήσιμα προσθετική.\\
Συνεπώς, σύμφωνα με το Θ. Καραθεοδωρή υπάρχει ένα μοναδικό μέτρο στον $((0,1],\mathcal{B}((0,1]))$ που είναι η προέκταση του $\mu_0$ στο $\mathcal{G}$ και το οποίο
ονομάζεται μέτρο \textlatin{Lebesgue}. (γενίκευση της Ευκλείδιας απόστασης)
\end{paradeigma}

\subsection{Ολοκλήρωση}
\label{sec:orgc505d9d}

\textbf{Μια παρατήρηση}: Ας εξετάσουμε τη συνάρτηση \(f:[0,1]\rightarrow \mathbb{R}\) η οποία ορίζεται ως
$$ f(x) = \begin{cases} 0, \quad \forall x \in \mathbb{Q}\cap[0,1]\\
                    1, \quad \forall x \in [0,1]\setminus \mathbb{R}
                    \end{cases}$$

Καθορίζουμε πρώτα μια διαμέριση \(0=x_0<x_1<\dots <x_n=1\) και μετά εξετάζουμε τα αθροίσματα \textlatin{Reiamman} για ρητούς αριθμούς \(\xi_i\) και παρατηρούμε
$$ \sum_{i=1}^n f(\xi_i)(x_i - x_{i-1})=0$$
Αν διαλέξω άρρητους \(xi_i\) τότε
$$ \sum_{i=1}^n f(\xi_i)(x_i - x_{i-1})=1$$

Συνεπώς είναι προφανές ότι αυτή η συνάρτηση δεν είναι \textlatin{Riemann} ολοκληρώσιμη.\\
Ωστόσο παρατηρώ ότι η \(f\) είναι η \(\mathbf{1}_{[0,1]\setminus \mathbb{Q}\). Ποιο είναι το μέτρο \textlatin{Lebesgue} του \(A=[0,1]\mathbb{Q}\); Γνωρίζουμε ότι οι ρητοί
ως αριθμήσιμη ένωση (ξένων) μονοσυνόλων είναι μετρήσιμοι, συνεπώς \(\mathbb{Q} = \sum_{i=1}^\infty \{a_i\} = 0\) αφού τα μονοσύνολα είναι σύνολα μέτρου 0, άρα έπεται ότι
το σύνολο των αρρήτων είναι:
$$\mu( [0,1] \setminus \mathbb{Q} ) = 1$$
Ουσιαστικά με τα παραπάνω συλλογιζόμαστε ότι:
$$ \int _{[0,1]} f(x) d\mu(x) = 1\cdot \mu([0,1]\setminus \mathbb{Q}) + 0 \cdot \mu([0,1]\cap \mathbb{Q}) = 1$$
\pagebreak

\section{Μάθημα 4}
\label{sec:orge8260a9}
Το ερώτημα είναι: \textbf{μπορώ να ολοκληρώσω τις απλές συναρτήσεις;}\\

Απλές συναρτήσεις \textlatin{(step functions)} είναι συναρτήσεις της μορφής
$$ f(x) = \sum_{i=1}^n c_i \mathbf{1}_{A_i} \qquad \text{όπου } \: A_i \cap A_j = \emptyset \text{ και } \bigcup_i A_i = \Omega$$

Στόχος μας είναι να ξεκινήσουεμ να κτίζουμε το ολοκλήρωμ από απλές συναρτήσεις και να γενικεύσουμε, καταλήγοντας στο ολοκλήρωμα γενικά για μετρήσιμες συναρτήσεις.

\subsection{Το ολοκλήρωμα \textlatin{Lebesgue}}
\label{sec:org8954775}
Θα ορίσουμε το ολοκλήρωμα \textlatin{Lebesgue} σε τρία βήματα.\\

Έστω \((\Omega,\mathcal{F})\) μετήσιμος χώρος και \(\mu:\mathcal{F}\rightarrow \mathbb{R} \cup \{+ \infty\}\) ένα μέτρο. Επίσης έστω
$$F: \Omega \rightarrow \mathbb{R}\cup\{+\infty\} \qquad \text{ μετρήσιμη συνάρτηση }$$

\textbf{Βήμα 1}\\
Θεωρώ ότι έχω \(f\geq 0\) απλές και μετρήσιμες συναρτήσεις της μορφής:
$$ f(x) = \sum_{i=1}^n c_i \mathbf{1}_{A_i} \qquad \text{όπου } \: A_i \cap A_j = \emptyset \text{ και } \bigcup_i A_i = \Omega$$
τότε ορζίουμε το ολοκλήρωμα \textlatin{Lebesgue} της \(f\) ως:
$$ \int_\Omega f d\mu = \sum_{i=1}^\inftu c_i \mu(A_i) \in [0,+\infty]$$
με την σύμβαση ότι στο ολοκλήρωμα \textlatin{Lebesgeue} \((0\cdot \infty = 0)\).\\

\textbf{Bήμα 2}\\
Τώρα θεωρούμε ότι έχουμε \(f\geq 0\) μετρήσιμες συναρτήσεις. Στην συνέχεια θα χρειασούμε το Θ. Μονότονης Σύγκλισης/\textlatin{Monotone Convergence Theorem}.

\begin{theorima}\textbf{Θεώρημα Μονότονης σύγκλησης.} Έστω $f\geq 0$ μετρήσιμη συνράτηση. Τότε μπορώ να βρώ (πάντοτε) μια ακολουθία μη-αρνητικών απλών συναρτήσεων (που όπως
είδαμε είναι μετρήσιμες), έστω $\{f_n\}_{n\geq 1}$, έτσι ώστε η $\{f_n\}_n$ να είναι αύξουσα ακολουθία ($f_n \subseteq f_{n+1}\: \forall n$) και
$$\lim_{n\rightarrow \infty} f_n(x) = f(x) \qquad \text{(pointwise - σημειακά)}$$
\end{theorima}
\pagebreak

Για να ορίσουμε το ολοκλήρωμα \textlatin{Lebesgue} για \(f\geq 0\) μετρήσιμες, χρησιμοποιούμε το Θεώρημα Μονότονης σύγκλισης:

$$ \Big( \int_\Omega \lim_{n\rightarrow \infty} f_n d\mu \Big) = \Big( \int_\Omega f d\mu \Big) = \Big( \lim_{n\rightarrow \infty} f_n d\mu \Big) $$

και τότε, χρησμοποιώντας το Θ. Μονότονης Σύγκλισης μπορούμε να αποδείξουμε ότι το
$$\Big( \lim_{n\rightarrow \infty} \int_\Omega f_n d\mu \Big)$$G\textsubscript{1}
είναι καλώς ορισμένο και δεν εξαρτάται από την επιλογή της ακολουθίας \(\{f_n\}_{n\geq 1}\).\\

\textbf{Βήμα 3}\\
Τέλος, έστω \(f\) μετρήσιμη συνάρτηση. Τότε μπορώ να γράψω την \(f\) χρησμοποιώντας το θετικό και το αρνητικό της μέρος, δηλαδή
$$ f = f^+ - f^-$$
όπου \(f^+(x)= \max \{ f(x), 0\}\) και \(f^-(x) = \max \{-f(x),0\}\).\\
Τότε το ολοκλήρωμα \textlatin{Lebesgue} ορίζεται ως
$$\int_\Omega f d\mu = \int_\Omega f^+ d\mu - \int_\Omega f^- d\mu $$
\pagebreak

\subsection{Ιδιότητες}
\label{sec:org817ae25}
\begin{enumerate}
\item Το ολοκλήρωμα \textlatin{Lebesgue} μιας μετρήσιμης συνάρτησης, όπου αυτό ορίζεται, είναι ένα στοιχείο του \([0,\infty]\).
\item Αν το μέτρο ενός έστω από τα \(A_i\) είναι ίσο με άπειρο, τότε το ολοκλήρωμα \textlatin{Lebesgue} παίρενει την τιμή \(+\infty\) (για κάθε \(c_i >0,\: i\geq 1\)).
\item Aν τα ολοκληρώματα \(\int_\Omega f^+ d\mu\) και \(\int_\Omega f^- d\mu\) παίρνουν την τιμ \(+\infty\) τότε το \(\int_\Omega fd\mu\) \textbf{δεν ορίζεται}.
\item Αν έχουμε ένα \textbf{φραγμένο διάστημα} \([a,b]\) με \(a,b \in \mathbb{R}\), το ολοκλήρωμα
$$\int_a^b f(x)d(x)$$
είναι καλως ορισμένο για \(f\) μετρήσιμη, τότε το ολοκλήρωμα Lebesgue
$$\int_[a,b]f d\mu $$
ισουται με το ολοκλήρωμα \textlatin{Riemann}.
\item Αν για μια μετρήσιμη συνάρτηση \(f\) υπάρχει το γενικευμένο ολ. \textlatin{Riemann}
$$\int_{-\infty}^\infty f(x)d<\inftyx \quad \text{ή}\quad \int_{-\infty}^\infty |f(x)|dx < \infty<\infty$$
τότε, το ολοκλήρωμα \textlatin{Lebesgue} \(\equiv\) \textlatin{Riemann.}
\item Μπορώ να έχω το γενικευμένο ολ. \textlatin{Riemann} αλλά όχι το αντίστοιχο \textlatin{Lebesgue}. (π.χ. \(f(x) = \frac{sinx}{x}\mathbf{1}_{\{x\neq 0\}}\))
\end{enumerate}

\subsection{Κύριες Ιδιότητες του ολοκληρώματος \textlatin{Lebesgue}:}
\label{sec:org4ce58d6}

\begin{itemize}
\item \textbf\{(Γραμμικότητα - \textlatin{Linearity})\}   $$\int_\mathbb{R} (c_1 f + c_2 g) d\mu = c_1 \int_\mathbb{R} fd\mu + c_2 \int_\mathbb{R} gd\mu$$.
\item \textbf\{(Ξένα Σύνολα - \textlatin{Disjoint Sets})\} Αν \(A,B\) είναι ξένα μεταξύ τους σύνολα, τότε
$$\int_{A\cup B} f d\mu = \int_A fd\mu + \int_B fd\mu$$
\item \textbf\{(Μονοτονία - \textlatin{Comparison})\} Αν \(f(x) \leq g(x)\) για κάθε \(x\in \mathbb{R}\), τότε
$$\int_\mathbb{R} f(x)d\mu(x) \leq \int_\mathbb{R} g(x)d\mu(x)$$.

\pagebreak
\end{itemize}
\section{Μάθημα 5}
\label{sec:org9bab349}

\subsubsection{Θεωρήματα Σύγκλισης}
\label{sec:org8925fd3}
\begin{theorima}[Μονότονης Σύγκλισης - \textlatin{Monotone Convergence Theorem (MCT)}.] Έστω $\{f_n\}_{n\geq 1}$ μια αύξουσα ακολουθία μετρήσιμων μη αρνητικών συναρτήσεων, οι οποίες
  συγκλίνουν σε μια συνάρτηση μετρήσιμη $f$, τότε
  $$ \int_\mathbb{R} f d\mu = \lim_{n\rightarrow \infty} \int_\mathbb{R} f_n d\mu $$
  όπου οι δύο πλευρές μπορούν να πάρουν την τιμή άπειρο.
\end{theorima}

\begin{theorima}[Λήμμα Fatou - \textlatin{Fatou Lemma (FL)}.] Έστω $\{f_n\}_{n\geq 1}$ μια ακολουθία μετρήσιμων, μη-αρνητικών συναρτήσεων, τότε
$$\int_\mathbb{R} \liminf_{n\rightarrow \infty} f_n d\mu  \leq \liminf_{n\rightarrow \infty} \int_\mathbb{R} f_n d\mu $$
\end{theorima}

\begin{proof}
Δημιουργώ την ακολουθία μετρήσιμων συναρτήσεων $\{g_n\}_{n\geq 1}$, όπου $g_k:= \inf_{n\geq k} f_n$.
Η $\{g_n\}$ συνεπώς είναι μια αύξουσα ακολουθία μη-αρνητικών μετρήσιμων συναρτήσεων, όπου
$$\lim_{n\rightarrow \infty} = \liminf_{n\rightarrow \infty} f_n$$

Συνεπώς, από \textlatin{MCT} έχουμε $\int_\mathbb{R} \lim_{k\rightarrow \infty} g_k d\mu = \lim_{k\rightarrow \infty} \int_\mathbb{R} g_k d\mu$, συνεπώς

\begin{align*}
\int_\mathbb{R} \liminf_{n\rightarrow \infty} f_n d\mu &= \lim_{k\rightarrow \infty} \int \inf_{n\geq k} f_n d\mu\\
&\leq \lim_{k\rightarrow \infty} \inf_{n\geq k} \int_\mathbb{R} f_n d\mu \\
( * ) \qquad &\leq \liminf_{n\rightarrow \infty} \int_\mathbb{R} f_n d\mu
\end{align*}

Όπου $(*)$ ισχύει διότι για κάθε $n\geq k, f_n \geq g_k$, συνεπώς $\int_\mathbb{R} f_n d\mu \geq \int_\mathbb{R} g_k d\mu$.
\end{proof}\\


Το λήμμα \textlatin{Fatou} μας λέει ότι μπορεί να έχω μια ακολουθία μετρήσιμων τ.μ. που να συκλίνει σε μια (μετρήσιμη) τ.μ. αλλά οι ροπές τους
(moments) να μην συγκλίνουν!!
\pagebreak

\begin{theorima}[Θεώρημα Κυριαρχημένης Σύγκλισης - \textlatin{(Lebesgue) Dominated Convergence Theorem (LDCT)}.] Έστω $\{f_n\}_{n\geq 1}$ μια ακολουθία ολοκληρώσιμων συναρτήσεων
η οποία συγκλίνει στην $f$ (σημειακή σύγκλιση - σύγκλιση σ.π/a.e.).\\
Αν υπάρχει μια ολοκληρώσιμη συνάρτηση $g\geq 0$ τέτοια ώστε $|f_n| \leq g$ (σχεδόν παντού) για κάθε $n\geq 1$, τότε η $f$ είναι
ολοκληρώσιμη και
$$\int_\mathbb{R} f d\mu = \lim_{n\rightarrow \infty} \int_\mathbb{R} f_n d\mu$$
\end{theorima}

\begin{proof}
Παρατηρούμε πρώτα ότι $|f_n - f| \leq |f_n| + |f| \leq g + g \leq 2g$ και ότι\\
$$\int_\mathbb{R} 2g d\mu = 2\int_{\mathbb{R}}g d\mu < \infty$$

Τώρα θα κάνουμε χρήση του FT. Έστω $h_n := 2g - |f_n - f|$, άρα η $\{h_n\}$ είναι μια μη-αρνητική ακολουθία μετρήσιμων συναρτήσεων, εφαρμόζω το λήμμα \textlatin{Fatou} και
$$\int_\mathbb{R} \liminf_{n\rightarrow \infty} h_n d\mu \leq \liminf_{n\rightarrow \infty} \int_\mathbb{R} h_n d\mu $$

Συνεπώς

$$ \cancel{\int_\mathbb{R} 2g d\mu} + \int_\mathbb{R} \liminf_{n\rightarrow \infty} (-|f_n - f|) d\mu \leq \cancel{\int_\mathbb{R} 2g d\mu} + \liminf_{n\rightarrow \infty} \int_\mathbb{R}(-|f_n -f|) d\mu$$

χρησιμοποιώντας ότι $-\limsup_{n\rightarrow \infty} -|f_n-f| = \liminf_{n\rightarrow \infty} |f_n -f|$ παίρνουμε

$$ - \int_\mathbb{R} \limsup_{n\rightarrow \infty} |f_n -f| d\mu \leq - \limsup_{n\rightarrow \infty} \int_\mathbb{R} |f_n-f|d\mu$$

Συνεπώς, πολλαπλασιάζοντας και τα δύο μέλη με $-1$, παίρνουμε
$$\limsup_{n\rightarrow \infty} \int_\mathbb{R} |f_n - f| d\mu \leq \int_\mathbb{R} |f_n-f| d\mu = 0$$
καθώς το $\limsup |f_n - f| = \lim |f_n- f| = 0$. Έχουμε δηλαδή
$$\lim_{n\rightarrow \infty} \int_\mathbb{R}|f_n - f| d\mu = 0$$
Ισχύει από comparison/monotonicity property ότι
$$\lim_{n\rightarrow \infty} | \int_\mathbb{R} fd\mu - \int_\mathbb{R} f_n d\mu| = \lim_{n\rightarrow \infty} |\int_\mathbb{R} (f_n -f) d\mu| \leq \lim_{n\rightarrow \infty} \int_\mathbb{R} |f_n-f| d\mu =0 $$
\end{proof}

\textbf{Σημείωση:} Τα παραπάνω τρία θεωρήματα σύκγλισης \textlatin{(MCT, FL, LDCT)}  ισχύουν σε σ-πεπερασμένους χώρος μέτρου \((\Omega, \mathcal{F}, \mu)\).\\

\begin{orismos} Έστω $(\Omega,\mathcal{F},\mathbb{P})$ ένας χώρος πιθανότητας. Τότε, μια συνάρτηση $X:\Omega \rightarrow \mathbb{R}$ ονομάζεται \textbf{τυχαία μεταβλητή} αν και μόνο αν
$$X^{-1}(B) \in \mathcal{F} \quad \forall B \in \mathcal{B}(\mathb{R})$$
\end{orismos}

\begin{orismos}
Έστω $\mu,\nu$ δύο μέτρα ορισμένα σε ένα μετρήσιμο χώρο $(\Omega, \mathcal{F})$. Αν για κάθε $A \in \mathcal{F}$ τ.ω. $\mu(A)=0$ τότε $\nu(A)=0$, τότε
λέμε ότι το $\nu$ είναι \textbf{απόλυτα συνεχές ως προς το} $\mu$ (absolutely continuous w.r.t $\mu$), και συμβολικά γράφουμε $\nu <\!< \mu$
\end{orismos}

\begin{theorima}[\textlatin{Radon-Nikodym}]
Έστω $\mu$ και $\nu$ δύο σ-πεπερασμένα μέτρα ορισμένα σε ένα μετρήσιμο χώρο $(\Omega, \mathcal{F})$ και $\nu <\!< \mu$. Τότε υπάρχει μοναδική (σχεδόν παντού)
μη αρνητική και ολοκληρώσιμη συνάρτηση $f$ στο $(\Omega, \mathcal{F}, \mu)$ τ.ω.
$$ \nu(A) = \int_A f d\mu  \quad \forall A \in \mathbb{F}$$
Χρησιμοποιούμε σαν συμβολισμό $d\nu = f d\mu$ \textlatin{(shorthand notation)} για να δηλώσουμε την σχέση μεταξύ των δύο μέτρων,
και η $f = \frac{d\nu}{d\mu}$ είναι γνωστή ως παράγωγος \textlatin{Radon-Nikodym (Radon-Nikodym derivative) }ή απλώς πυκνότητα \textlatin{(density)} του $\nu$ ως προς το $\mu$.
\end{theorima}

\pagebreak

\section{Μάθημα 6}
\label{sec:orgc84f7b9}

\textbf{Παρατήρηση}: Στο Θεώρημα \textlatin{Radon Nikodym} αυστηρά δεν έχουμε ορίσει κάποια παράγωγο μέτρου σε σχέση με κάποιο άλλο μέτρο, και ο συμβολισμός της πυκνότητας


$$f = \frac{d\nu}{d\mu}\qquad \text{ή} \qquad d\nu = fd\mu$$
ωστόσο, αν δούμε την απόδειξη του Θεωρήματος, αν έχουμε τρία μέτρα \(\nu,\mu,\rho\) και πυκνότητες \(g=\frac{d\nu}{d\mu}\) και \(f=\frac{d\mu}{d\rho}\) μπορούμε να πούμε \(gf=\frac{d\nu}{d\rho}\), δηλαδή συμβολικά:
$$ \frac{d\nu}{d\mu} \frac{d\mu}{d\rho} = \frac{d\nu}{d\rho}$$
όπου πρακτικά ``απλοποιούμε'' το κλάσμα. Υπενθυμίζουμε ότι δεν έχουμε παραγώγους και όλα αυτά τα κάνουμε συμβολικά αλλά παίρνουμε έγκυρα αποτελέσματα.\\

\textbf{Σύνδεση/εφαρμογή με τα χρηματοοικονομικά μαθηματικα}: Προσπαθούμε να βρούμε ισοδύναμα μέτρα πιθανότητας ως προς το ``φυσικό'' μέτρο πιθανότητας έτσι ώστε να δημιουργήσουμε στο νέο μέτρο \textlatin{martingale} (δίκαια παιχνίδια). Υπό το
νέο μέτρο, όταν γίνεται η αποτίμηση να μην υπάρχουν ευκαιρίες για \textlatin{arbitrage}.\\

\begin{orismos}
Έστω $X:\Omega \Rightarrow \mathbb{R}$ τ.μ. σε ένα χώρο πιθανότητας $(\Omega,\mathcal{F},\mathbb{R})$. Η απεικόνιση $\mathbb{F}_X:\mathcal{B}(\mathbb{R}) \rightarrow [0,1]$ που ορίζεται ως
$$ \mathbb{F}_X(B):= \mathbb{P}(X^{-1}(B))] = \mathbb{P}( \{ \omega \in \Omega\: : \: X(\omega) \in B\} ) \in \[0,1]\qquad \forall B \in \mathcal{B}(\mathbb{R})$$
και ονομάζεται \textbf{κατανομή} της $X$ \textlatin{(distribution or law of the r.v.} $X$).\\
\end{orismos}

\textbf{Σημείωση}: Στην θέση του μετρήσιμου χώρου \((\mathbb{R},\mathcal{B}(\mathbb{R}))\) μπορεί να χρησιμοποιηθεί κάποιος άλλος μετρήσιμος χώρος \((S,\mathcal{H})\).\\

\pagebreak
\begin{protasi}
Η κατανομή $\mathbb{F}_X$ είναι μέτρο πιθανότητας στον $(\mathbb{R},\mathcal{B})$.
\end{protasi}

\begin{proof} Αρκεί να δείξουμε ότι ικανοποιεί τις ιδιότητες ενός μέτρου πιθανότητας.
\begin{enumerate}
\item $\mathbb{F}_X (B) \in [0,1]$ για κάθε $B\in \mathcal{B}$.
\item $\mathbb{F}_X (\mathbb{R}) = \mathbb{P}[X^{-1}(\mathbb{R})] = \mathbb{P}[\Omega]$. Ομοίως δείχνω ότι $\mathbb{F}_X (\emptyset) = \mathbb{P}[X^{-1}(\emptyset)] = \mathbb{P}[\emptyset] = 0$.
\item Αν τα $A_1,A_2,\dots \in \mathcal{B}$ είναι ξένα μεταξύ τους ανά δύο, τότε:
$$\mathbb{F}_X (\cup_i A_i) = \mathbb{P}[X^{-1}(\cup_{i} A_i)] = \mathbb{P}[\cup_i X^{-1}(A_i)]$$
και παρατηρώ ότι τα $X^{-1}(A_i)$ είναι ξένα μεταξύ τους ανά δύο, οπότε χρησιμοποιώ την αρ. προσθετικότητα του $\mathbb{P}$ και παίρνω
$$\mathbb{F}_X(\cup_i A_i) = \sum_{i=1}^\infty \mathbb{P}(X^{-1}(A_i)) = \sum_{i=1}^\infty \mathbb{F}_X(A_i)$$
\end{enumerate}
\end{proof}

\begin{itemize}
\item Οι συναρτήσεις κατανομής ορίζονται από την σχέση \(\mathbb{F}_X(x) := \mathbb{F}((-\infty,x]) = \mathbb{P}(X\leq x)\)
\item Οι συνάρτηση κατανομής είναι μοναδική (να γίνει απόδειξη).
\item Το αντίστροφο επίσης ισχύει, δηλαδη: για κάθε συνάρτηση κατανομής \(F\) υπάρχει μοναδική κατανομή \(\mathbb{F}\) τ.ω. η σχέση που έχουμε πιο πάνω να ικανοποιείται, δηλαδή
$$F(x) = \mathbb{F}((-\infty,x])$$
να ικανοποιείται.
\end{itemize}


Εμείς επιθυμούμε να ορίσουμε την \(\mathbb{E}[X] = \int_\Omega X d\mathbb{P}\), και θα χρησιμοποιήσουμε το πιο κάτω θεώρημα έτσι ώστε να μην απαιτείται
ο υπολογισμός του ολοκληρώματος \textlatin{Lebesgue} μέσω απλών συναρτήσεων, αλλά μέσω ολοκληρωμάτων \textlatin{Riemann} με τα οποία είμαστε εξοικειωμένοι.

  \begin{theorima}(\textbf{αλλαγής μεταβλητής}) Έστω $X:\Omega \rightarrow \mathbb{R}$ μια τ.μ. που ορίζεται στον χ.π. $(\Omega,\mathcal{F},\mathbb{P})$ και $g$ μια (Borel) μετρήσιμη συνάρτηση.
    Τότε
$$ \int_\Omega g(X(\omega)) d\mathbb{P}(\omega) = \int_\mathbb{R} g(x)d\mathbb{F}_X(x)$$
δηλαδή αντί να κάνω τον υπολογισμό στον $(\Omega, \mathcal{F}, \mathbb{P})$ τον κάνω στον $(\mathbb{R},\mathcal{B},\mathbb{F}_X)$
  \end{theorima}

\begin{proof} Κάνουμε την απόδειξη σε τρία βήματα.\\

\begin{enumerate}
\item Αν $g(x) = \sum_{i=1}^n c_i\mathbf{1}_{A_i}(x)$, όπου $c_i \in \mathbb{R}, A_i \cap A_j = \emptyset$ και $\cup_{i=1}^n = \mathbb{R}$, τότε
\begin{align*}
 \int_\Omega g(X(\omega)) d\mathbb{P}(\omega) &= \int_\Omega \sum_{i=1}^n \mathbf{1}_{A_i}(X(\omega))d\mathbb{P}(\omega) = \sum_{i=1}^nc_i\int_\Omega \mathbf{1}_{A_i}(X(\omega))d\mathbb{P}(\omega)\\
 &= \sum_{i=1}^nc_i\int_{\{ \omega\in\Omega : X(\omega) \in A_i\}} X(\omega) \mathbb{P}(\omega) = \sum_{i=1}^n c_i \int_{X^{-1}(A_i)} 1 \mathbb{P}\\
&= \sum_{i=1}^n c_i \mathbb{P}(X^{-1}(A_i)) = \sum_{i=1}^n c_i \mathbb{F}_X(A_i)\\
&= \sum_{i=1}^n c_i \int_{A_i} 1 d\mathbb{F}_X(x)= \sum_{i=1}^n c_i \int_{\mathbb{R}} \mathbf{1}_{A_i} (x)d\mathbb{F}_X(x)\\
&= \int_{\mathbb{R}} \sum_{i=1}^n c_i \mathbf{1}_{A_i} (x)d\mathbb{F}_X(x)= \int_{\mathbb{R}} g(x) d\mathbb{F}_X(x)
\end{align*}
συνεπώς έχουμε δείξει ότι ισχύει για απλές συναρτήσεις.
\item  Αν η $g$ είναι (Borel) μετρήσιμη συνάρτηση η οποία παίρνει μη αρνητικές τιμές. Τότε, υπάρχει αύξουσα ακολουθία μετρήσιμων συναρτήσεων $\{g_n\}_{n\in\mathbb{N}}$ ώστε $\lim_{n\rightarrow \infty} g_n(x) = g(x)$ σ.π.
Χρησιμοποιώντας το Θεώρημα Μονότονης Σύγκλισης παρατηρούμε ότι:
$$\int_\Omega g(X(\omega))d\mathbb{P}(\omega) = \lim_{n\rightarrow \infty} \int_\Omega g_n(X(\omega)) d\mathbb{P}$$
Επίσης, το όριο
$$ \lim_{n\rightarrow \infty} \int_\mathbb{R} g_n(x) d\mathbb{F}_X(x) = \int_\mathbb{R} g(x)d\mathbb{F}_X(x)$$
\item Τέλος, αν η $g$ είναι μια (Borel) μετρήσιμη συνάρτηση, τότε χρησιμοποιούμε την σχέση
$$g = g^+ - g^-$$
για να καταλήξουμε στο επιθυμητό αποτέλεσμα.
\end{enumerate}
\end{proof}

Και πάλι, στην θέση του \((\mathbb{R},\mathcal{B})\) μπορούμε να έχουμε τον \((S,\mathcal{H})\) η \(g\) θα πρέπει να είναι \(\mathcal{H}\) μετρήσιμη και η \(X\) θα πηγαίνει από το \(\Omega\) στο \(\mathbb{R}\).\\

\begin{orismos} Αν υπάρχει \textlatin{Borel} μετρήσιμη συνάρτηση $f_X:\mathcal{B}(\mathbb{R}) \rightarrow \mathbb{R}$ έτσι ώστε $\forall B \in \mathcal{B}$
$$\mathbf{F}_X(B) = \int_B f_X(x) d\mu(x)$$
όπου $\mu$ είναι το μέτρο \textlatin{Lebesgue}, τότε λέμε ότι η $X$ είναι τυχαία μεταβλητή με \textbf{συνεχή κατανομή} και η $f_X$ ονομάζεται \textbf{πυκνότητα} (density) της $X$ (αλλά και της κατανομής $\mathbb{F}_X$).
\end{orismos}

\section{Μάθημα 7}
\label{sec:orgd55053a}

\begin{orismos}
Έστω $(\Omega,\mathcal{F},\mathbb{P})$ ένας χώρος πιθανότητας και $X:\Omega \rightarrow S$ μια τυχαία μεταβλητή στον χώρο αυτό που παίρνει διακριτές τιμές $x_1, x_2, \dots \in S$, όπου $(S,\mathcal{H})$ ένας μετρήσιμος χώρος. Τότε λέμε ότι η $X$ έχει διακριτή κατανομή με μάζα $\mathbb{P}(X=x_i) = \mathbb{P}(\{\omega \in \Omega : X(\omega) = x_i\})$.
\end{orismos}

\begin{orismos}
Έστω $X$ μια τ.μ. στον χώρος πιθανότητας $(\Omega, \mathcal{F}, \mathbb{P})$. Ορίζουμε ως μέση τιμή (expectation) της $X$ το ολοκλήρωμα $\int_\Omega X d\mathbb{P}$, δηλαδή $\mathbb{E}[X] = \int_\Omega X d\mathbb{P}$
\end{orismos}

\begin{orismos}
Έστω $X:\Omega \rightarrow \mathbb{R}$ μια τ.μ. στον $(\Omega, \mathcal{F},\mathbb{P})$ με $\mathbb{E}[|X|^2]<\infty$. Ορίζουμε την διασπορά (variance) της $X$ ως το ολοκλήρωμα $\int_\Omega |X-\mathbb{E}[X]|^2 d\mathbb{P}$, δηλαδή
$$ Var(X) = \mathbb{E}[|X-\mathbb{E}[X]|^2] $$
\end{orismos}

\textbf{Διακριτές Τυχαίες Μεταβλητές:} \(X(\omega) = \sum_{i}x_i\mathbf{1}_{A_i}(\omega), \: x_i \in \mathbb{R}.\: A_i \cap A_j = \emptyset\) για \(i\neq j\) με \(A_i \in \mathcal{F}\) για κάθε \(i\geq 1\).

\begin{align*}
\mathbb{E}[X] &= \int_\Omega X d\mathbb{P} = \int_\Omega \sum_i x_i \mathbf{1}_{A_i} = \sum_i \int_\Omega x_i \mathbf{1}_{A_i} d\mathbb{P} = \sum_i \int_{A_i} x_i d\mathbb{P}\\
&= \sum_i x_i \int_{A_i} d\mathbb{P} = \sum_i x_i \mathbb{P}(A_i) = \sum_i
\end{align*}

\begin{theorima}[Ανισότητα \textlatin{Markov}]
 Έστω $X:\Omega \rightarrow \mathbb{R}$ μια τυχαία μεταβλητή στον χώρο πιθανότητας $(\Omega,\mathcal{F},\mathbb{P})$ η οποία παίρνει μη-αρνητικές τιμές και $c>0$. Τότε
$$ \mathbb{P}(X\geq c) \leq \frac{\mathbb{E}[X]}{c}  \qquad \mathbb{E}[X] \geq \mathbb{E}[x\mathbf{1}_{\{x \geq c\}}] = c \mathbb{P}(X\geq c)$$
\end{theorima}

\begin{theorima}[Ανισότητα \textlatin{Chebyshev}]
Έστω $X:\Omega \rightarrow \mathbb{R}$ μια τ.μ. στον $(\Omega, \mathcal{F}, \mathbb{P})$ με $\mathbb{E}[|X|]<\infty$ και $c>0$. Τότε
$$\mathbb{P}(|X-\mathbb{E}[X]|\geq c) \leq \frac{Var(X)}{c^2} $$
\end{theorima}

\begin{theorima}[Ανισότητα \textlatin{Jensen}]
Έστω $X:\Omega \rightarrow \mathbb{R}$ μια τ.μ. στον χ.π. $(\Omega,\mathcal{F},\mathbb{P})$ και $\phi:\mathbb{R} \rightarrow \mathbb{R}$ μια κυρτή συνάρτηση και επίσης $\mathbb{E}[X] <\infty$. Τότε
$$\mathbb{E}[\phi(X)] \geq \phi \big( \mathbb{E}[X]\big)$$
\end{theorima}

\subsection{Χώροι \(L^p\)}
\label{sec:orgb80a8d2}

Χώροι \(L^p, p>0\): Έστω \((S,\mathcal{H},\mu)\) ένας σ-πεπερασμένος χώρος μέτρου. Το σύνολο όλων των μετρήσιμων συναρτήσεων \(f:S \rightarrow V\), όπου \(V,\mathcal{G}\) μετρήσιμος χώρος, οι οποίες έχουν την ιδιότητα
$$ \Big( \int_S |f|_V ^p d\mu \Big)^{1/p}  <\infty $$

όπου \(|\cdot|_V\) η νόρμα που παράγεται από τον \(V\).

Για εμάς \(S=\Omega, \mathcal{H}=\mathcal{F}, \mu = \mathbb{P}\) και θέλουμε όλες τις τ.μ. τ.ω.
$$||X||_p = \Big( \int_\Omega |X|^p d\mathbb{P} \Big)^{1/p} < \infty $$
όταν \(p>1\) έχουμε την λεγόμενη \(L^p\) νόρμα.

\subsection{Σύγκλιση}
\label{sec:org70c1ffa}

\begin{orismos}
Έστω $\{X_n\}_{n\geq 1}$ μια ακολουθία τυχαίων μεταβλητών σε να χώρο πιθανότητας $(\Omega, \mathcal{F},\mathbb{P})$. Τότε μέμε ότι
\begin{enumerate}
\item η ακολουθία συγκλίνει σε μια τυχαία μεταβλητή $X$ \emph{σχεδόν βέβαια} (ή με πιθανότητα 1) και γράφουμε $X_n \stackrel{\text{σ.β.}}{\rightarrow} X$, αν
$$P(\lim_{n\rightarrow \infty} X_n = X) = 1 $$
δηλαδή αν $P( \{\omega \in \OmegaL \lim_{n\rightarrow \infty} X_n(\omega) = X(\omega\} ) = 1$
\item η ακολουθία συγκλίνει σε μια τυχαία μεταβλητή $X$ \emph{κατά πιθανότητα} \textlatin{(in probability)} και γράφουμε $X_n \stackrel{\mathbb{P}}{\rightarrow} X$ αν
$$\lim_{n\rightarrow \infty}\mathbb{P}(|X_n - X| > \epsilon) = 0 \qquad \forall \epsilon >0$$
\item η ακολουθία συγκίνει σε μια τυχαία μεταβλητή $X$ \emph{κατά κατανομή} \textlatin{(in distribution)} και γράφουμε $X_n \stackrel{d}{\rightarrow} X$ αν
$$ \lim_{n\rightarrow \infty} \underbrace{\mathbb{P}(X_n \leq x)}_{F_{X_n}(x)} = \underbrace{\mathbb{P}(X\leq x)}_{F_X(x)}$$
σε κάθε σημείο συνέχειας $x$ της συνάρτησης κατανομής $F_X$
\item η ακολουθία συγκλίνει σε μια τυχαία μεταβλητή $X$ \emph{στον $L^p$} και γράφουμε $X_n \stackrel{L^p}{\rightarrow} X$ αν
$$\lim_{n\rightarrow \infty} \mathbb{E}[|X_n - X|^p]=0$$
\end{enumerate}
\end{orismos}

Ισχύει το ακόλουθο σχήμα που συνδέει τις πιο πάνω συγκλίσεις
\begin{align*}
&\text{σ.β.} \\
&\Downarrow \\
\text{στον } L^p  \Rightarrow \text{ κατά}&\text{ πιθανότητα } \\
& \Downarrow \\
\text{κατά}&\text{ κατανομή}
\end{align*}

\section{Μάθημα 8}
\label{sec:org49f7f29}
\pagebreak

\section{Μάθημα 9}
\label{sec:orgf154557}

Υποθέτω για όλα τα παρακάτω ότι υπάρχει ένας χώρος πιθανότητας \((\Omega,\mathcal{F},\mathbb{P})\).

\begin{orismos}
    Έστω $A,B \in \mathcal{F}$ και $\mathbb{P}(A) \neq 0$, τότε ορίζουμε τη \textbf{δεσμευμένη πιθανότητα} του $A$ δοθέντος/δεδομένου του $B$
    ως εξής
$$ \mathbb{P}(A|B) := \frac{\mathbb{P}(A \cap B)}{\mathbb{P}(B)}$$
\end{orismos}

\begin{orismos}
    Τα ενδεχόμενα $A,B \in \mathcal{F}$ λέμε ότι είναι \textbf{ανεξάρτητα} (μεταξύ τους) αν
$$ \mathbb{P}(A \cap B) = \mathbb{P}(A) \mathbb{P}(B) $$
\end{orismos}

\begin{orismos}
    Δύο τυχαίες μεταβλητές $X:\Omega \rightarrow \mathbb{R}$ και $Y:\Omega \rightarrow \mathbb{R}$ ονομάζονται
    \textbf{ανεξάρτητες} αν για οποιαδήποτε $A,B \in \mathbcal{B}(\mathbb{R}$)$ τα ενδεχόμενα $X^{-1}(A)$ και $Y^{-1}(B)$ είναι ανεξάρτητα.
\end{oritmos}

\begin{orismos}
    Δύο σ-άλγεβρες $\mathcal{F}_1, \mathcal{F}_2 \subset \mathcal{F}$, ονομάζονται \textbf{ανεξάρτητες} αν οποιαδήποτε ενδεχόμενα
    $A\in \mathcal{F}_1$ και $B \in \mathcal{F}_2$ έχουμε ότι είναι ανεξάρτητας.
\end{orismos}

\begin{paradeigma}
    Δύο τυχαίες μεταβλητές $X$ και $Y$ είναι ανεξάρτητες αν και μόνο αν οι παραγόμενες σ-άλγεβρες $\sigma(X), \sigma(Y)$ είναι ανεξάρτητες.
\end{paradeigma}

\begin{orismos}
    Η ελάχιστη σ-άλγεβρα που περιέχει όλες τις προ-εικόνες (\textlatin{pre-images}) $X^{-1}(A), \: \forall A \in \mathcal{B}{\mathbb{R}}$ μιας τυχαίας μεταβλητής $X$,
    ονομάζεται \textbf{σ-άλγεβρα παραγόμενη από την} $X$, και συμβολίζεται με $\sigma(X)$. \\
\end{orismos}

\textbf{Σημεωίση:} Ο ορισμός επεκτείνεται με φυσικό τρόπο σε πεπερασμένο πλήθος τ.μ. \(X_1, \dots, X_n\) για την δημιουργία παραγόμενης σ-άλγεβρας \(\sigma(X_1,\dots,X_n)\) από
αυτές τις τυχαίες μεταβλητές.

\begin{paradeigma}
    Αν $X$ και $Y$ είναι ανεξάρτητες τ.μ. τότε
 $$ \forall x,y, \in \mathbb{R} \quad \mathbb{P}(X\leq x, Y\leq y) = \mathbb{P}(X\leq x) \mathbb{P}(Y\leq y)$$
και συνεπώς
$$F_{X,Y}(x,y) = F_X(x) F_Y(y)$$
\end{paradeigma}

\begin{orismos}
    Έστω $X$ μια ολοκληρώσιμη τυχαία μεταβλητή και $B \in \mathcal{F}$ με $\mathbb{P}(B) \neq 0$. Τότε ορίζουμε τη \textbf{δεσμευμένη μέση τιμή} της $X$ δοθέντος του
    ενδεχομένου $B$ ως
$$ \mathbb{E}[X|B] = \frac{1}{\mathbb{P}(B)}\int_B Xd\mathbb{P} = \frac{\mathbb{E}[X\mathbf{1}_B]}{\mathbb{E}[\mathbf{1}_B]} $$
\end{orismos}

\textbf{Παρατήρηση:} Αν θέσω \(X = \mathbf{1}_A\) τότε εύκολα βλέπουμε ότι
$$\mathbb{E}[\mathbf{1}_A|B] = \frac{1}{\mathbb{P}(B)}\int_B \mathbf{1}_A d \mathbb{P} = \frac{1}{\mathbb{P}(B)} \int_{A\cap B} d\mathbb{P} = \frac{P(A\cap B)}{\mathbb{P}}$$

\textbf{Σημείωση} Όταν δύο τ.μ. \(X\) και \(Y\) είναι ανεξάρτητες, τότε \(\mathbb{E}[XY] = \mathbb{E}[X] \mathbb{E}[Y]\) (αφήνεται ως άσκηση)


\begin{orismos}
    Έστω $X$ μια $L^1$ τυχαία μεταβλητή, δηλαδή $\mathbb{E}[|X|]<\infty$. Τότε ορίζουμε την \textbf{δεσμευμένη μέση τιμή} της $X$ δοθείσης της
    διακριτής τυχαίας μεταβλητής

  $$Y = \sum_{i\geq 1} y_i \mathbf{1}_{A_i} \qquad \text{όπου } A_i = \{Y = y_i \} \quad \forall i\geq 1$$

 ως την τυχαία μεταβλητή $\mathbb{E}[X|Y]$ τ.ω.

    $$\mathbb{E}[X|Y]  = \sum_i \mathbb{E}[X|\{Y = y_i\} ]\mathbf{1}_{ \{Y=y_i\} }$$
\end{orismos}


\begin{paradeigma}
Έστω $(\Omega,\mathcal{F},\mathbb{P})$ ένας χώρος πιθανότητας όπου $\Omega = [0,1],\: \mathcal{F} = \mathcal{B}([0,1])$ και $\mathbb{P}$ το μέτρο
\textlatin{Lebesgue} στο $[0,1]$.\\
Έστω επίησης οι τ.μ. $X,Y: \Omega \rightarrow \mathbb{R}$ όπου
$$ X(\omega) = 2\omega^2 , \quad Y(\omega) = \begin{cases}
                                                    1, \quad & \forall \omega \in [0,1/3) =A_1\\
                                                    2, \quad & \forall \omega \in [1/3,2/3)= A_2\\
                                                    0, \quad & \forall \omega \in [2/3, 1] = A_3
                                            \end{cases} $$

Παρατηρούμε ότι η $Y$ έχει διακριτή κατανομή και ότι $\{ \omega \in \Omega : Y(\omega) = 1\} = \{ Y = 1\} = [0,1/3)$. Ομοίως $\{Y=2\} = [1/3,2/3)$ και $\{Y=0\}= [2/3,1]$
Aλλιώς μπορούμε να δούμε το παραπάνω μέσω της παραγόμενης σ-άλγεβρας της $Y$

$$ \sigma(Y) = \{ A_1, A_2, A_3, A_1 \cup A_2, A_1 \cup A_3, a_2 \cup A_3, \Omega, \emptyset \} $$

Άρα, η τ.μ. $\mathbb{E}[X|Y] : \Omega \rightarrow \mathbb{R}$ ορίζεται ως


$$
\mathbb{E}[X|Y](\omega) = \begin{cases}
    \mathbb{E}[X|[0,\frac{1}{3}] = \frac{1}{\mathbb{P}([0,1/3))}\mathbb{E}[X\mathbf{1}_{[0,1/3)}] = \frac{1}{3} \int_0^{1/3}2x^2 dx = \frac{2}{27}&, \quad  \forall \omega \in [0,1/3)\\
    \mathbb{E}[X|[\frac{1}{3},\frac{2}{3}] = \frac{14}{27}&, \quad  \forall \omega \in [0,1/3)\\
    \mathbb{E}[X|[\frac{2}{3}],1]] = \frac{38}{27}&, \quad  \forall \omega \in [0,1/3)\\
\end{cases}$$

\end{paradeigma}


\pagebreak

\section{Μάθημα 10}
\label{sec:org1f898cc}

\begin{enumerate}
\item Έστω ένας χ.π. \((\Omega, \mathcal{F},\mathbb{P})\) και \(\mathcal{G}\subset \mathcal{F}\). Τότε για την \(\mathbb{E}[X|\mathcal{G}]\) έχουμε
\begin{enumerate}
\item \(\mathbb{E}[X|\mathcal{G}]\) είναι \(\mathcal{G}-\text{μετρήσιμη}\).
\item \(\forall A \in \mathcal{G}\) έχουμε την ιδιότητα του \textlatin{partial averaging}:
$$ \int_A X d\mathbb{P} = \int_A \mathbb{E}[X|Y]d\mathbb{P} \quad \text{ή ισοδύναμα} \quad \mathbb{E}[X\mathbf{1}_A] = \mathbb{E}\Big[ \mathbb{E}[X|Y]\mathbf{1}_A \Big]$$
\end{enumerate}
\item Μπορούμε να γράψουμε \(\mathbb{E}[X|\sigma(Y)]\) αντί για \(\mathbb{E}[X|Y]\) και αντίστροφα.
\item \(\mathbb{P}(A|\mathcal{G}) = \mathbb{E}[\mathbf{1}_A|\mathcal{G}]\) \textlatin{(consistency)}.
\end{enumerate}


\begin{paradeigma}
Η μέση τιμή $\mathbb{E}[X]<\infty $ μιας τυχαίας μεταβλητής $X\in L^1$ είναι έξνας αριθμός που μπορεί να θεωρηθεί ως τετριμμένη τυχαία μεταβλητή μιας και μπορούμε να γράωουμε $\mathbb{E}[X|\mathcal{F}_0] = \mathbb{E}[X]$ όπου
η $\mathcal{F}_0 = \{\emptyset, \Omega\}$ είναι η τετεριμμένη σ-άλγεβρα.
\end{paradeigma}

\textbf{Σημείωση:} Μια διαφορετική προσέγγιση είναι να δούμε την μέση τιμή ως την καλύτερη πρόβλεψη/εκτιμήτρια \textlatin{(prediction/estimator)} που έχουμε για την τυχαία μεταβλητή (που εξετάζουμε κάθε φορά),
όταν ορίζουμε την ιδιότητα \emph{καλύτερη} μέσω της ελαχιστοποίησης του μέσου τετραγωνικού σφάλματος.

\begin{limma}{($L^2-\text{προβολής}$ ή καλύτερης πρόβλεψης).} Αν $\mathbb{E}[X^2]<\infty$, τότε η δεσμευμένη μέση τιμή $\mathbb{E}[X|\mathcal{F}_0]$ ελαχιστοποέι την $\mathbb{E}[(X-Y)^2]$ για όλες της $\mathcal{F}_0-\text{μετρήσιμες}$ τυχαίες μεταβλητές $Y\in L^2$.
\end{limma}


\textbf{Απόδειξη}
    Παρατηρούμε ότι
\begin{align*}
\mathbb{E}[(X-Y)^2]
&=\mathbb{E}[(X-\mathbb{E}[X|\mathcak{F}_0] + \mathbb{E}[X|\mathcal{F}_0] - Y)^2]\\
&=\mathbb{E}\Big[ (X-\mathbb{E}[X|\mathcal{F}_0])^2 \Big] + 2 \underbrace{\mathbb{E}\Big[(X-\mathbb{E}[X|\mathcal{F}_0]) (\mathbb{E} [ X|\mathcal{F}_0] - Y) \Big]}_{( * )} + \mathbb{E}\Big[(\mathbb{E}[X|\mathcal{F}_0])^2\Big]
\end{align*}

Στο \(( * )\) έχουμε από την \(\mathcal{F}_0-\text{μετρησιμότητα}\) των \(\mathbb{E}[X|\mathcal{F}_0]\) και \(Y\) (που τις καθιστά σταθερές , αφού η \(\mathcal{F}_0\) είναι η τετριμμένη σ-άλγεβρα):


$$ \text{ (*) } = (\mathbb{E}[X|\mathcal{F}_0] - Y ) \underbrace{\mathbb{E}[X - \mathbb{E}[X|\mathcal{F}_0]]}}_{=\mathbb{E}[X] - \mathbb{E}[X]} = 0 $$


\textbf{Άσκηση:}

$$ \mathbb{E} \Big[ (X- \mathbb{E}[X|\mathcal{G}])^2\Big] = \min \{ \mathbb{E}[(X-Z)^2] \: : \: \text{όπου η } Z \in \mathcal{G}, Z \in L^2 \}$$

\subsection{Θεμελιώδεις ιδιότητες της δεσμευμένης μέσης τιμής.}
\label{sec:org8169e81}

\begin{enumerate}
\item \textlatin{(linarity)}. Για κάθε \(a,b \in \mathbb{R}, \mathcal{G} \subset \mathcal{F}\) και \(X,Y : \Omega \rightarrow \mathbb{R}^2\), με \(X,Y \in L^1\) τότε
$$ \mathbb{E}[ aX + bY | \mathcal{G}] = a \mathbb{E}[X|\mathcal{G}] + b \mathbb{E}[Y|\mathcal{G}]$$
\item \textlatin{(positivity)}. Για κάθε \(X:\Omega \rightarrow \mathbb{R}^2\), με \(X \in L^1\) τυχαία μεταβλητή τότε
$$ X \geq 0 \: \text{σ.β.} \Rightarrow \mathbb{E}[X|\mathcal{G}] \geq 0 \: \text{σ.β.} $$
\item \textlatin{(taking out what's known)}. Αν \(X: \Omega \rightarrow \mathbB{R}^d\) είναι μια \(\mathcal{G}-\text{μετρήσιμη}\), \(L^1\) τυχαία μεταβλητή, τότε για κάθε \(Y:\Omega \rightarrow \mathbb{R}^d, Y \in L^1\) έχουμε
$$ \mathbb{E}[XY|\mathcal{G}] = X \mathbb{E}[Y|\mathcal{G}] $$
\item \textlatin{(tower property)}. Αν οι σ-λάγεβρες \(\mathcal{G}_1, \mathcal{G}_2\) είναι τέτοιες ώστε \(\mathcal{G}_1 \subset \mathcal{G}_2 \subset \mathcal{F}\), τότε για κάθε \(X\in L^1\) τ.μ. έχουμε
$$ \mathbb{E}\Big[ \mathbb{E}[X|\mathcal{G}_2 | \mathcal{G}_1 \Big] = \mathbb{E}[X|\mathcal{G}_1] \quad \text{σ.β.}$$
\item Αν \(X\) μια \(L^1\) τ.μ. ανεξάρτητη από την σ-άλγεβρα \(\mathcal{G}\subset \mathcal{F}\), τότε
$$\mathbb{E}[X|\mathcal{G}] = \mathbb{E}[X] \quad \text{σ.β.}$$
\item \textlatin{(Jensen's inequality)}. Αν \(X \in L^1\) και \(\phi :\mathb{R} \rightarrow \mathbb{R}\) μια κυρτή συνάρτηση, τότε
$$ \phi \Big( \mathbb{E}[X|\mathbb{G}] \Big) \leq \mathbb{E}\Big[ \phi(X) | \mathcal{G} \Big]$$
\end{enumerate}


\textbf{Απόδειξη της ιδιότητας 5:} Αν η \(X\) είναι ανεξάρτητη της σ-άλβεγρας \(\mathcal{G}\), τότε η \(X\) είναι ανεξάρτητη από την δείκτρια \(\mathbf{1}_A\) οποιουδήποτε συνόλου \(A \in \mathcal{G}\).
Έτσι, για κάθε \(A \in \mathcal{G}\) έχουμε από \textlatin{partial averaging property}
$$ \int_A X d\mathbb{P} = \int_\Omega A \mathbf{1}_A d\mathbb{P} = \mathbb{E}[X\mathbf{1}_A] = \mathbb{E}[X]\mathbb{E}[\mathbf{1}_A] = \mathbb{E}[X]\int_A 1 d\mathbb{P} = \int_A \mathbb{E}[X] d\mathbb{P}$$
Επίσης, η \(\mathbb{E}[X]\)  είναι \(\mathcal{G}-\text{μετρήσιμη}\) άρα \(\mathbb{E}[X|\mathcal{G}] = \mathbb{E}[X] \: \text{σ.β.}\)

\pagebreak

\textbf{ΠΑΡΑΤΗΡΗΣΗ}: Το Θεώρημα μονότονης σύγκλισης, το Λήμμα \textlatin{Fatou}, και το Θεώρημα κυριαρχημένης σύγκλισης ισχύουν και αν χρησιμοποιήσουμε δεσμευμένες μέσες τιμές, δηλαδή εάν αντικαταστήσουμε τα ολοκληρώματα
\textlatin{Lebesgue} \(\int \cdot d\mu\) με \(\mathbb{E}[\cdot | \mathcal{G}]\), όπου \(\mathcal{G}\) μια σ-άλγεβρα.\\


\textbf{ΠΑΡΑΤΗΡΗΣΗ}: Έστω μια σ-αγεβρα \(\mathcal{G}\subset \mathcal{F}\) και \(X\) μια τ.μ. ανεξάρτητη της \(\mathcal{G}\). Επίσης, έστω μια συβάρτηση \(h: \mathbb{R}^d \times \mathbb{R}^d \rightarrow \mathbb{R}^d, d \geq 1\) η οποία είναι
\textlatin{Borel} μετρήσιμη και τ.ω. \(\mathbb{E}[|h(X,Y)|] < \infty\) για κάποια \(\mathcal{G}-\text{μετρήσιμη}\) τυχαία μεταβητή \(Y\). Τότε
$$ \mathbb{E}\big[ h(X,Y) | \mathcal{G} \big]  = f(Y)$$
όπου \(f(y) = \mathbb{E}\big[h(X,y)\big]\) για κάθε \(y \in \mathbb{R}^d\)

\pagebreak

\section{Μάθημα 11}
\label{sec:org6d772c8}
\subsection{Εισαγωγή στις Στοχαστικές Διαφορικές Εξισώσεις}
\label{sec:orgb4d6fe8}

Οι διαφορικές εξισώσεις \textlatin{(ordinary differential equations -ODEs)} της μορφής

$$ \frac{dX_t}{dt} = b(X_t) $$

χρησιμοποιούνται συχνά για να περιγράψουν την ανέλιξη μιας ποσότητας \(X_t\), όταν η διαφορά \(\Delta X_t := X_{t+\Delta t} - X_t\), κατά την διάρκεις ενός πολύ μικρού χρονικού διαστήματος \([t, t+\Delta t]\) είναι προσεγγιστικά ίση με \(b(X_t)\Delta t\), όπου \(b\) είναι μια συνάρτηση. Πολλές φορές χρησιμοποιούμε και
την μορφή του ολοκληρώματος, δηλ.

$$ X_t = X_0 + \int_0^t b(X_s)ds $$

Παρ'ολα αυτά, οι διαφορικές εξισώσεις δεν είναι ο κατάλληλος τρόπος για να περιγράψουμε την ανέλιξη της \(X_t\) όταν οι αλλαγές της \(X_t\)
επηρεάζονται από τυχαία φαινόμενα.

Η κίνηση \textlatin{Brown (Brownian motion)} χρησιμοποιείται πολλές φορές για να περιγράψει αυτή την επίδραση των τυχαίων φαινομένων.
Όμως, οι τροχιές \textlatin{(trajectories)} της κίνησης \textlatin{Brown} είναι σχεδόν παντού μη παραγωγίσιμες \textlatin{(nowhere differentiable)}, οπότε αν \(W:= \{W_t\}_{t\geq 0}\) συμβολίζει την κίνηση \textlatin{Brown}, θα θέλαμε να δώσουμε νόημα σε εκφράσεις
όπως

$$ \Delta X_t \approx b(X_t) \Delta_t + \sigma (X_t) \Delta W_t$$

ή όπως

$$X_t = X_0 + \int_0^t b(X_s) ds + \underbrace{\int_0^t \sigma(X_s) dW_s}_{(*)} $$

Το δεύτερο ολοκλήρωμα \((*)\)  δεν μπορεί να οριστεί σύμφωνα με τη κλασική ανάλυση.
\pagebreak

\section{Μάθημα 12}
\label{sec:org2d6ffb0}

$$dX_t = b(X_t)dt + \text{"noise"}$$

Θεωρούμε την κίνηση \textbf{Brown} \(W = \{W_t\}_{t\geq 0}\) και στον \((\Omega, \mathcal{F},\mathbb{P})\) θα θέλαμε να έχουμε κάτι της μορφής
s




\$\$\text{"} dX\textsubscript{t} = b(X\textsubscript{t})dt + \(\sigma\) (X\textsubscript{t}) dW\textsubscript{t} \text{"} \qquad \(\omega\) \(\in\) \(\Omega\)\$

και θα θέλαμε να έχουμε μια καλώς ορισμένη ολοκληρωτική μορφή

$$ X_t = X_0 + \int_0^t b(X_s)ds + \underbrace{\int_0^t \sigma(X_s)dW(s)}_{(*)} $$
\subsection{Κίνηση \textlatin{Brown}  / Διαδικασία \textlatin{Wiener}}
\label{sec:orgf96855f}

Έστω \((\Omega, \mathcal{F}, \mathbb{P})\) ένας χώρος πιθανότητας και \(F = \{F_t\}_{t\geq 0}\)  μια διύληση (filtratrion) τέτοια ώστε η \(\mathcal{F}_0\) να περιέχει όλα τα σύνολα της \(\mathcal{F}\)
που έχουν πιθανότητα μηδέν (\(\mathbb{P}-\textlatin{null sets}\)).

\begin{orismos}
Μια στοχαστική διαδικασία $\{W_t\}_{t\geq 0}$ ονομάζεται κίνηση \textlatin{Brown} (διαδικασία \textlatin{Wiener}) αν
\begin{enumerate}
    \item έχει ανεξάρτητες προσαυξήσεις (\textlatin{independent increments}), δηλαδή
$$ W_{t_1} - W_{t_0} , \dots , W_{t_n} - W_{t_{n-1}} $$
είναι ανεξάρτητες τ.μ. για οποιαδήποτε $0\leq t_1 \leq t_2 \leq \dots \lew t_n $ και $n\geq 2$.
\item $W_{t} - W_s \sim \mathcal{N}(0, t-s)$ για όλα τα $t,s \in \mathbb{R}$ τέτοια ώστε $0\leq s \leq t$
\item είναι μια συνεχς στοχαστική διαδικασία, δηλαδή οι τροχίες
$$ W_{\cdot}(\omega), \quad \omega \in \Omega$$
είναι συνεχείες συναρτήσεις του $t$. ($t \mapsto W_t(\omega)$ είναι συνεχείς συναρτήσεις για σχεδόν όλα τα $\omega \in \Omega$).
\item $W_0 = 0$
\end{enumerate}
\end{orismos}

\textbf{\textbf{Σημείωση:}} Πολλές φορές ο ισοδύναμος ορισμός χρησιμοποιείται, όταν το 1. δίνεται ως εξής:
\(W_t - W_s\) είναι ανεξάρτητη από \(\mathcal{G} = \sigma(W_r, r\leq s)\) για όλα τα \(t,s\) τ.ω. \(0\leq s \leq t\).\\

\textbf{Σημείωση:} Χρησιμοποιείται ο όρος \textlatin{Wiener martingale} ως προς \(\{F_t\}_{t\geq 0}\) αν \(W:= \{W_t\}_{t\geq 0}\) είναι προσαρμοσμένη (\textlatin{adapted}) στην \(\{F_t\}_{t\geq 0}\) και \(W_t - W_s\) είναι
ανεξάρτητη της \(\mathcal{F}_s\) όταν \(0\leq s \leq t\)


\begin{paradeigma}
Έστω $\{W_t\}_{t\geq 0}$.

\begin{itemize}
\item $\{W_t\}_{t\geq 0}$ είναιθ προσαρμοσμένη στην $\{\mathcal{F}_t\}$, δηλ η $W_t$ είναι $\mathcal{F}-\text{μετρήσιμη}$ για κάθε $t\geq 0$.
\item $W_t - W_0 \sim \mathcal{N}(0,t-0)$ δηλαδή $W_t \sim \mathcal{N}(0,t)$ για κάθε $t\geq 0$.
\item $\mathbb{E}[|W_t|] = \frac{1}{\sqrt{2\pi t}} \int_{-\infty}^\infty |x|e^{-\frac{x^2}{2t}} dx < \infty, \: \forall t >0 $.
\item Και εη $W_t$ είναι \textlatin{martingale} καθώς:
$$\mathbb{E}[W_t|\mathcal{F}_s] = \mathbb{E}[W_t - W_s + W_s |\mathcal{F}_s] = \mathbb{E}[W_t - W_s | \mathcal{F}_s] + \mathbb{E}[W_s|\mathcal{F}_s] = 0 + W_s = W_s , \quad \forall 0 \leq s < t$$
\end{itemize}
\end{paradeigma}

\begin{paradeigma}
$X:= \{X_t\}_{t\geq 0}$, όπου $X_t = W_t^2 - t, \quad \forall t \geq 0$.
\begin{itemize}
\item (Μετρησιμότητα) Η $X_t$ είναι $\mathcal{F}_t -\text{μετρήσιμη}$ για κάθε $t \geq 0$.
\item (Ολοκληρωσιμότητα) $\mathbb{E}[|X_t|] = \mathbb{E}[|W_t^2 - t|] \leq \mathbb{E}[W_t^2] + \mathbb{E}[t] = 2t < \infty, \:\forall t\geq 0$
\item Και
\begin{align*} \mathbb{E}[X_t|\mathcal{F}_s] &= \mathbb{E}[W_t^2 - t | \mathcal{F}_s] = \mathbb{E}[W_t^2 | \mathcal{F}] - t \\
& = \mathbb{E}[(W_t - W_s + W_s)^2 | \mathcal{F}_s] - t \\
&= \mathbb{E}[(W_t-W_s)^2|\mathcal{F}_s] +  2\mathbb{E}[(W_t - W_s)W_s | \mathcal{F}_s] + \mathbb{E}[W_s^2 | \mathcal{F}_s] -t \\
&= \mathbb{E}[(W_t - W_s)^2 | \mathcal{F}_s] + W_s\underbrace{\mathbb{E}[W_t - W_s]}_{=0} + W_s^2 - t\\
&= t-s+W_s^2 - t = W_s^2 -t
\end{align*}
\end{itemize}
\end{paradeigma}

\begin{paradeigma}
$Y = \{Y_t\}_{t\geq 0}$ όπου $Y_t := e^{-\frac{c^2}{2}t c W_t}, \: \forall t\geq 0$ και $c >0$
\begin{itemize}
\item (Μετρησιμότητα) Η $Y_t$ είναι $\mathcal{F}_t - \text{μετρήσιμη}, \: \forall t \geq 0$ (σύνθεση συνεχούς συνάρτησης με την $W_t$)
\item (Ολοκληρωσιμότητα)
\begin{align*}\mathbb{E}[Y_t] &= \mathbb{E}[e^{-\frac{c^2}{2} + c W_t}] \\
&= \frac{1}{\sqrt{2 \pi t}}\int_{-\infty}^\infty e^{-\frac{c^2}{2}t + cx} e^{-\frac{x^2}{2t}} dx \\
&= \frac{1}{\sqrt{2\pi t}} \int_{-\infty}^\infty e^{-\frac{c^2t^2 - 2ctx + x^2}{2t}}dx\\
&= \frac{1}{\sqrt{2\pi t}} \int_{-\infty}^\infty e^{-\frac{(x-ct)^2}{2t}}dx = 1
\end{align*}
\item Και για την τελευταία ιδιότητα

\begin{align*}
\mathbb{E}[Y_t | \mathcal{F}_s] &= \mathbb{E}\big[ e^{-\frac{c}{2}t + c W_t} | \mathcal{F}_s\big] \\
&= e^{-\frac{c}{2}t} \mathbb{E} \big[ e^{cW_t} | \mathcal{F}_s \big] \\
&= e^{-\frac{c^2}{2}t}e^{cW_s} \mathbb{E}\big[ e^{c(W_t - W_s)} | \mathcal{F}_s \big] \\
&= e^{-\frac{c^2}{2}t} e^{cW_s} \mathbb{E}\big[ e^{c(W_t-W_s)}\big] = e^{-\frac{c^2}{2}t}e^{cW_s}e^{\frac{c^2}{2}(t-s)}\\
&= e^{-\frac{c^2}{2}s + cW_s}, \quad \forall 0 \leq s < t
\end{align*}

αφού παρατηρώ από τον προηγούμενο υπολογισμό ότι
$$ \mathbb{E}\Big[ e^{-\frac{c^2}{2}t + c W_t}\Big] = 1 \Rightarrow  \mathbb{E}\Big[ e^{cW_t} \Big] = e^{\frac{c^2}{2}}
\begin{itemize}
\end{paradeigma}

\begin{orismos} Ονομάζουμε $d-\text{διάστατη}$ κίνηση \textlatin{Brown}, όπου $d \geq 1$ ένας φυσικός αριθμός, τη στοχαστική διαδικασία $W = \{W_t\}_{t\geq 0}$ όπου
$$ W_t := (W_t^{(1)}, \dots, W_t^{(d)})^T $$
και $W^{(i)}:= \{W_t^{(i)}\}_{t\geq 0}$ είναι μια τυπική κίνηση \textlatin{Brown} για κάθε $i\in\{1,\dots, d\}$

\end{orismos}
\end{document}
